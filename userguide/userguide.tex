\newcommand{\documentTitle}{Usage Guide for the Annie Rose Quantum Networking Framework}
\newcommand{\documentVersion}{Version 0.1}

\documentclass[letterpaper,twoside=false]{scrbook}
\usepackage[english]{babel}
\usepackage{titlesec} % Used to customize chapters, sections, and subsections
\usepackage{scrpage2} % Used for page header/footers
\usepackage{hyperref} % Used for urls

\newcommand{\code}[1]{\texttt{#1}}

\pagestyle{scrheadings}
\addtokomafont{pagenumber}{\oldstylenums}
\clearscrheadfoot
\lohead[\documentTitle]{}
\lehead[\documentTitle]{}
\rofoot[\center\thepage]{}
\refoot[\center\thepage]{}

\begin{document}
\title{\documentTitle}
\date{\documentVersion}
\maketitle
\tableofcontents

\chapter{Preface}

This project aims to provide a platform to simulate heterogeneous computer networks comprised of classical and quantum communication channels. The guiding design principles include software-defined communication using near-real time processing of streaming data, separation of concerns into different networking layers, in addition to object oriented software layers.

After reading this guide you will be familiar with the overall architecture of the network simulator, be able to run basic simulations, and make basic modifications to simulations.

\chapter{Authors and Acknowledgments}

\begin{tabular}{ l l }
	Travis Humble & humblets@ornl.gov \\
	Ronald Sadlier & sadlierrj@ornl.gov \\
	Ryan Prout & proutrc@ornl.gov
\end{tabular}

%The authors would like to acknowledge 

The numerical simulator implements a modified CHP which was originally written  by Scott Aarsonson. Visit \url{www.scottaaronson.com/chp} for more information.

\chapter{Component Overview}

	\begin{description}
		\item[mininet] \hfill \\
		We use \href{www.mininet.org}{mininet} to create virtual networks consisting of nodes and switches.
		
		\item[orsoqs] \hfill \\
		This is a small user-level library used for communication with armish-fireplace.
		
		\item[armish-fireplace] \hfill \\
		This package is the client middleware, connecting client applications with either hardware or the quantum simulation backend. 
		
		\item[eldispacho] \hfill \\
		The dispatcher is responsible for the quantum network. Any process on the network that interactions with quantum information must do so through the dispatcher. The result of this constraint is that the entirety of the quantum network state is within the dispatcher. Noise models are computed here, allowing for the possibility of complex noise models involving other actors on the network.
		
		\item[dspy] \hfill \\
		Dspy is the \href{www.wireshark.org}{wireshark} for quantum communications. It allows you to examine the quantum network in detail and allows you to access the internal state of the quantum simulator.
		
		\item[sabot] \hfill \\
		Sabot contains the quantum numerical simulator in addition to components mostly for networking, and lexical analysis. Interactions with Sabot are through a ZMQ socket with messages formatted using JSON. Quantum states are stored here, with the simulation abstracted away from those interacting with the instance. The network state as a whole may not be stored here, but the quantum states of physical systems within the network that have not been measured are stored here. Currently, we use CHP (originally written by Scott Aaronson) as the numerical simulator.
		
		\item[Open vSwitch] \hfill \\
		Open vSwitch is a popular open-source software defined networking switch. We've extended it to communicate and configure our simulated quantum switch. Our extensions don't affect the compilation or installation procedure.
		
	\end{description}

\chapter{Installation}
	\section{Platforms}
	
	All code is written to be run on x$86\_64$ processors, in particular Intel 64. AMD64 has not been tested thoroughly, although some testing of software components on ARMv7-a hardware has yielded positive results.
	
	\section{Environment}

		Our environment consists of a standard mainstream Linux distribution with a Bash shell and the proper dependencies installed, along with minor configuration tweaks.

	\section{Download}
 		
 		We use Git to manage our source repositories, so to download the software is to clone our repositories onto your local machine. Simply clone the main repository with the --recursive flag:
 		
 		\code{
 		\$ git clone https://github.com/ornl-qci/annie-rose.git --recursive
 		}
 
\chapter{Usage Examples}
 	\section{Launching Example Network}
 		See examples/3host for an example of a 3 host quantum network. See the README file within the directory for information on launching.
 	
\chapter{Known Issues}
\chapter{References}

\end{document}
